% $Id: paint.tex,v 1.1 2005/05/04 17:00:49 greve Exp $
\documentclass[10pt]{article}
\usepackage{amsmath}
%\usepackage{draftcopy}

%%%%%%%%%% set margins %%%%%%%%%%%%%
\addtolength{\textwidth}{1in}
\addtolength{\oddsidemargin}{-0.5in}
\addtolength{\textheight}{.75in}
\addtolength{\topmargin}{-.50in}

%%%%%%%%%%%%%%%%%%%%%%%%%%%%%%%%%%%%%%%%%%%%%%%%%%%%%%%%%%%%%%%%%
%%%%%%%%%%%%%%%%%%%%%%% begin document %%%%%%%%%%%%%%%%%%%%%%%%%%
%%%%%%%%%%%%%%%%%%%%%%%%%%%%%%%%%%%%%%%%%%%%%%%%%%%%%%%%%%%%%%%%%
\begin{document}

\begin{Large}
\noindent {\bf paint} \\
\end{Large}

\noindent 
\begin{verbatim}
Comments or questions: analysis-bugs@nmr.mgh.harvard.edu\\
$Id: paint.tex,v 1.1 2005/05/04 17:00:49 greve Exp $
\end{verbatim}

\section{Introduction}
{\bf paint} is a program for mapping voxel intensities from a
functional volume onto a surface.

\section{Usage}
Typing paint at the command line without any options will give the
following message:\\ 

\begin{small}
\begin{verbatim}
Usage: paint input\_file hemi surf output\_file -options
  input\_file - eg, stem\_\%03d.bfloat
  surf - surface name (eg, orig, smoothwm, etc)
  hemi - lh or rh
  output\_file - eg, stem-lh.w 
  options:
    -imageoffset n - zero-based frame number to paint <0>
    -nslices  n - number of input slices
    -regdat   fname - name of register file <register.dat>
    -dmax  dist - distance (mm) to project along normal <0>
    -dstep stepsize - size (mm) of projection step <0.25>
\end{verbatim}
\end{small}

\section{Command-line Arguments}

\noindent
{\bf input\_file}: this is the input stem of the functional input 
volume in bfile format.  It will take the form {\em
stem\_\%03d.bfloat}. \\

\noindent
{\bf surf}: name of the surface upon which to paint (eg, orig,
smoothwm, etc).\\

\noindent
{\bf hemi}: hemisphere string (lh or rh).\\

\noindent
{\bf output\_file}: name of the file where the results will be stored.
This file must have a ``w'' extension.\\

\noindent
{\bf -imageoffset }: the functional data may have many different
time-points (also known as planes or frames) for each voxel, however,
the point output can only represent one of these frames.  The number
of frames in a functional volume is indicated by the third item in the
functional volume's header file.  The image offset allows the user to
choose which frame to paint.\\

\noindent
{\bf -nslices}:  number of slice files in the input functional volume.\\

\noindent
{\bf -regdat}: this allows the user to specify a registration file.
The default is to use the one in the current directory.\\

\section{Example}

Consider the case where there is a functional volume with stem {\em
pavf} with 32 slice files.  Running the unix {\em ls} in the directory
will show bfiles of the form {\em pavf\_000.bfloat, pavf\_001.bfloat,
\ldots, pavf\_031.bfloat} and their corresponding header files (with
{\em .hdr} extension.


\end{document}