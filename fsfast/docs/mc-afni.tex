% $Id: mc-afni.tex,v 1.1 2005/05/04 17:00:48 greve Exp $
\documentclass[10pt]{article}
\usepackage{amsmath}
%\usepackage{draftcopy}

%%%%%%%%%% set margins %%%%%%%%%%%%%
\addtolength{\textwidth}{1in}
\addtolength{\oddsidemargin}{-0.5in}
\addtolength{\textheight}{.75in}
\addtolength{\topmargin}{-.50in}

%%%%%%%%%%%%%%%%%%%%%%%%%%%%%%%%%%%%%%%%%%%%%%%%%%%%%%%%%%%%%%%%%
%%%%%%%%%%%%%%%%%%%%%%% begin document %%%%%%%%%%%%%%%%%%%%%%%%%%
%%%%%%%%%%%%%%%%%%%%%%%%%%%%%%%%%%%%%%%%%%%%%%%%%%%%%%%%%%%%%%%%%
\begin{document}

\begin{Large}
\noindent {\bf mc-afni} \\
\end{Large}

\noindent 
\begin{verbatim}
Comments or questions: analysis-bugs@nmr.mgh.harvard.edu\\
$Id: mc-afni.tex,v 1.1 2005/05/04 17:00:48 greve Exp $
\end{verbatim}

\section{Introduction}
{\bf mc-afni} is a front-end for running the AFNI motion correction
(see http://varda.biophysics.mcw.edu/~cox/ for more information about
AFNI). You must have the AFNI software package installed on your
computer and the binaries in your path.  It converts from bfile format
to AFNI format then uses AFNI's 3dvolreg to align the functional
volume to a target volume.  Algorithm used by 3dvolreg: iterated
linearized weighted least squares to make each sub-brick as like as
possible to the base brick.  This method is useful for finding SMALL
MOTIONS ONLY.  \\

\section{Usage}
Typing mc-afni at the command-line without any options will give the
following message:\\ 

\begin{small}
\begin{verbatim}
USAGE: mc-afni
Options:
   -i stem : input  volume 
   -o stem : output volume 
   -t stem : target (input volume)
   -toff offset  : target volume offset   (0)
   -session dir  : afni session directory (/tmp)
   -prefix  name : afni prefix            (procid)
   -scriptonly scriptname   : don't run, just generate a script
   -version : print version and exit
\end{verbatim}
\end{small}

\section{Command-line Arguments}

\noindent
{\bf -i stem}: (Required) stem of the input functional volume for a single run.
It is assumed that the data are stored in {\em bfile format}. \\

\noindent
{\bf -o stem}: (Required) stem of the output motion-corrected functional
volume. This script will also produce a file called {\em stem.mcdat}
in which the AFNI motion parameters are stored.  There are 9 columns:
{\em n roll pitch yaw dS dL dP rmsold rmsnew}, where n is the
functional index, roll is the rotation about the I-S axis, pitch is
the rotation about the R-L axis, yaw is the rotation about the A-P
axis, dS is displacement in the Superior direction, dL is the
displacement in the Left direction, dP is the displacement in the
Posterior direction, rmsold is the RMS difference between input brick
and base brick, rmsnew is RMS difference between output brick and base
brick.  All rotations are in degrees CCW, and all displacements are in
mm. See the AFNI documentation for 3dvolreg for more information.\\

\noindent
{\bf -t stem}: (Optional) stem of the target volume.  The target is the image to
which AFNI attempts to register the input volume.  The default is the
first image of the input volume.

\noindent
{\bf -toff offset}: (Optional) register to the $n^{th}$ image in the target
volume ($n = offset$), where the first image has offset 0.\\

\noindent
{\bf -session dir}: (Optional) location in which to place the temporary AFNI
files. The default is to put it in /tmp which should be local to the
computer executing the program.  This can speed up the run time
significantly as compared to when the session directory is someplace
else across the network. All temprorary files are deleted by {\bf
mc-afni}. \\

\noindent
{\bf -prefix name}: (Optional) name to give the temporary AFNI files.\\

\noindent
{\bf -scriptonly scriptname}: (Optional) this does not do anything yet.\\

\noindent
{\bf -version}: print {\bf mc-afni} version and exit.\\

\section{Examples}

\subsection{Single Run Motion Correction}

Consider the case where there is a functional volume with stem
$func$ with 16 slices, each with 120 time points.  To motion correct
this volume, run:
\begin{verbatim}
% mc-afni -i func -o func_mc
\end{verbatim}
This will create a new volume with stem {\em func\_mc}.  It will also
create a file called {\em func\_mc.mcdat} with the motion parameters.
Note that this implicity aligned all of the images with the first
image.  Note that the above invocation is equivalent to 
\begin{verbatim}
% mc-afni -i func -o func_mc -t func -toff 0
\end{verbatim}
To align with the $20^{th}$ image, run:
\begin{verbatim}
% mc-afni -i func -o func_mc -toff 19
\end{verbatim}

\subsection{Mulitple Run Motion Correction}

When there are multiple runs, you will want to align the images in
both runs to a single (target) image.  Consider two functional
volumes, $func1$ and $func2$, and we want to align both volumes to the
third image in $func1$.  Then run:
\begin{verbatim}
% mc-afni -i func1 -o func1_mc -toff 2
% mc-afni -i func2 -o func2_mc -t func1 -toff 2
\end{verbatim}
Again, this will create motion parameter files {\em func1\_mc.mcdat}
and {\em func2\_mc.mcdat}.


\end{document}