% $Id: mtp-repack.tex,v 1.1 2005/05/04 17:00:49 greve Exp $
\documentclass[10pt]{article}
\usepackage{amsmath}
%\usepackage{draftcopy}

%%%%%%%%%% set margins %%%%%%%%%%%%%
\addtolength{\textwidth}{1in}
\addtolength{\oddsidemargin}{-0.5in}
\addtolength{\textheight}{.75in}
\addtolength{\topmargin}{-.50in}

%%%%%%%%%%%%%%%%%%%%%%%%%%%%%%%%%%%%%%%%%%%%%%%%%%%%%%%%%%%%%%%%%
%%%%%%%%%%%%%%%%%%%%%%% begin document %%%%%%%%%%%%%%%%%%%%%%%%%%
%%%%%%%%%%%%%%%%%%%%%%%%%%%%%%%%%%%%%%%%%%%%%%%%%%%%%%%%%%%%%%%%%
\begin{document}

\begin{Large}
\noindent {\bf mtp-repack} \\
\end{Large}

\section{Introduction}
{\bf mtp-repack} is a program for filling in ``missing'' time points
in a functional volume.  This has been a problem with the GE LX system
which could not reconstruct images fast enough and so would drop them.
This program fills in the missing time points by interpolating between
the two adjacent time points. Missing time points may be obtained by
running {\bf aws\_plot\_times} (only on SunOS). {\bf mtp-repack} will
run on any unix platform with matlab 5.2 or higher installed. Note
that the missing time points must be accounted for in any statistical
analysis (even with interpolation).\\

\section{Questions, Comments, and Bug Reports}
Send questions, comments, and bug reports to
analysis-bugs@nmr.mgh.harvard.edu.  Make sure to include sufficient
information so that the question can be answered or the problem can be
solved. If possible, include the log file that is created by {\bf
mtp-repack}.\\

\section{Usage}
Typing mtp-repack at the command-line without any options will give the
following message:\\ 

\begin{small}
\begin{verbatim}
USAGE: mtp-repack  repack data with missing timepoints
Options:
   -i instem      : stem of input volume
   -o outstem     : stem of output volume
   -mtp filename  : name of file with missing timepoints
   -TR  TR        : TR in seconds
   -tpx filename  : name of tp exclude file (outstem.tpexclude)
   -umask umask   : set unix file permission mask
   -monly  mfile  : do not run, just create a matlab file
   -version       : print version and exit
\end{verbatim}
\end{small}

\section{Command-line Arguments}

\noindent
{\bf -i instem}: stem of the input functional volume for a single run.
This is the data with missing time points.\\

\noindent
{\bf -o outstem}: stem of the output functional volume with the
missing time points inserted.\\

\noindent
{\bf -mtp filename}: name of file in which information about the
missing time points is stored. See Section \ref{mtp.sec}.\\

\noindent
{\bf -TR TR}: the TR in seconds.\\

\noindent
{\bf -tpx filename}: time point exclusion file in a format usable by
{\bf selxavg}. The default is {\em outstem.tpexclude}\\

\noindent
{\bf -monly}: only generate the matlab file which would accomplish the
analysis but do not actually execute it.  This is mainly good for
debugging purposes.\\

\section{The Missing Time Point (MTP) File}
\label{mtp.sec}

The information about which time points are missing is stored in the
{\em MTP File} which is passed to {\bf mtp-repack} using the -mtp
flag.  The MTP file is a text file with a list of indices. The indices
can have spaces, tabs, or new lines between them. Each index indicates
the time point in the input data set before which a missing image
should be inserted (where the first time point is 1, not 0).  For
example, if the MTP file stored the number ``20'', this would indicate
that the an image should be inserted between the 19th and 20th images.
The inserted image would become the 20th image in the new volume.  The
20th image in the old volume would become the 21st image in the new
volume, etc.  \\

This way of coding the missing time points is consistent with the
program {\bf aws\_plot\_times}.  This program will produce a plot in
which spikes indicate missing time points.  If this program is used,
then one simply needs to create an MTP file and enter the indices
corresponding to these spikes.  A sample of the output from {\bf
aws\_plot\_times} is given below.  One can see that the value in the
second column is usually around 2400.  How ever, at indices 20 and
38, the value spikes up to 4864 indicating that an image was dropped.
One would then create an MTP file with the values ``20'' and ``38'' in
them.  

\begin{verbatim}
    2 2432.000000 007/I.041
    3 2432.000000 007/I.061
    4 2432.000000 007/I.081
    5 2432.000000 007/I.101
    6 2432.000000 007/I.121
    7 2432.000000 007/I.141
    8 2432.000000 007/I.161
    9 2432.000000 007/I.181
   10 2432.000000 007/I.201
   11 2432.000000 007/I.221
   12 2432.000000 007/I.241
   13 2432.000000 007/I.261
   14 2560.000000 007/I.281
   15 2304.000000 007/I.301
   16 2560.000000 007/I.321
   17 2432.000000 007/I.341
   18 2432.000000 007/I.361
   19 2432.000000 007/I.381
   20 4864.000000 007/I.401
   21 2432.000000 007/I.421
   22 2432.000000 007/I.441
   23 2432.000000 007/I.461
   24 2432.000000 007/I.481
   25 2432.000000 007/I.501
   26 2432.000000 007/I.521
   27 2432.000000 007/I.541
   28 2432.000000 007/I.561
   29 2432.000000 007/I.581
   30 2432.000000 007/I.601
   31 2432.000000 007/I.621
   32 2432.000000 007/I.641
   33 2432.000000 007/I.661
   34 2432.000000 007/I.681
   35 2432.000000 007/I.701
   36 2432.000000 007/I.721
   37 2432.000000 007/I.741
   38 4864.000000 007/I.761
   39 2432.000000 007/I.781
   40 2432.000000 007/I.801
\end{verbatim}



\end{document}