% $Id: isxavg-re.tex,v 1.1 2005/05/04 17:00:48 greve Exp $
\documentclass[10pt]{article}
\usepackage{amsmath}
%\usepackage{draftcopy}

%%%%%%%%%% set margins %%%%%%%%%%%%%
\addtolength{\textwidth}{1in}
\addtolength{\oddsidemargin}{-0.5in}
\addtolength{\textheight}{.75in}
\addtolength{\topmargin}{-.50in}

%%%%%%%%%%%%%%%%%%%%%%%%%%%%%%%%%%%%%%%%%%%%%%%%%%%%%%%%%%%%%%%%%
%%%%%%%%%%%%%%%%%%%%%%% begin document %%%%%%%%%%%%%%%%%%%%%%%%%%
%%%%%%%%%%%%%%%%%%%%%%%%%%%%%%%%%%%%%%%%%%%%%%%%%%%%%%%%%%%%%%%%%
\begin{document}

\begin{Large}
\noindent {\bf isxavg-re} \\
\end{Large}

\noindent 
\begin{verbatim}
Comments or questions: analysis-bugs@nmr.mgh.harvard.edu\\
$Id: isxavg-re.tex,v 1.1 2005/05/04 17:00:48 greve Exp $
\end{verbatim}

\section{Introduction}

{\bf isxavg-re} is a program for intersubject averaging using a {\em
random-effects model} (see isxavg-fe for fixed effects model).  It
actually performs statistical analysis as well.  It works by computing
one number for each subject (at each voxel) based on the contrast
matrix.  This yields a set of numbers (at each voxel), one for each
subject.  The average and standard deviation of the set are computed
and used to compute a t-statistic the siginficance of $H_0: t=0$ is
computed using $N-1$ degrees of freedom where $N$ is the number of
subjects.  When {\em jackknifing} is turned on, a new set of $N$
numbers is computed by successively averaging $N-1$ subjects while
excluding a different subject each time.  This reduces the effects of
small sample sizes on the correctness of the siginficance levels.
Jackknifing does increase the time to process data.  It is highly
recommended and the default in isxavg-re.  Note that isxavg-re can be
run on any data set, not just the output of {\em selxavg}.

\section{Usage}
Typing isxavg-re at the command-line without any options will give the
following message:\\ 

\begin{small}
\begin{verbatim}
USAGE: isxavg-re [-options] -cmtx cmtxfile -i instem1 -i instem2 -o outstem
   cmtxfile - contrast matrix file generated by mkcontrast
   instem   - prefix of .bfloat selxavg files
   outstem  - prefix of .bfloat output files
Options:
   -voltype              : <auto>, selxavg, selavg, bvolume
   -firstslice <int>     : first slice to process <0>
   -nslices <int>        : number of slices to process <auto>
   -format string        : <log10>, ln, raw
   -nojackknife          : turn off jackknifing <on> 
   -invert               : compute 1-p
   -monly  mfile         : dont run, just create a matlab file
   -version              : print version and exit
\end{verbatim}
\end{small}

\section{Command-line Arguments}

\noindent
{\bf -o cmtxfile}: contrast matrix file created by calling {\em
mkcontrast}.  The contrast matrix determines which statistical test
will be run.\\

\noindent
{\bf -i instem1}: stem of the volume in which the results of the
selective averaging for the first subject have been stored (see
selxavg).  There must be at least two input volumes, each preceded by
a {\em -i} flag.\\

\noindent
{\bf -o outstem}: stem of the output volume.  The format will be the
same as the input volumes, and isxavg-re will also produce appropriate
dof, dat, and hdr files.\\

\noindent
{\bf -voltype}: This is the format of the type of data that will be
averaged together.\\

\noindent
{\bf -firstslice int}: first {\em anatomical} slice to process (usually 0).
This should not be confused with the first {\em functional} slice.\\

\noindent
{\bf -nslices int}: total number of {\em anatomical} slices to process.\\

\noindent
{\bf -format}: transformation to apply to the significances stored in
the output file. Options: ln (natural log), log10, and raw. The
default is log10.\\

\noindent
{\bf -nojackknife}: do not jackknife the averages.\\

\noindent
{\bf -invert}: report one minus the signficance value instead of the
signficance value. \\

\noindent
{\bf -monly}: only generate the matlab file which would accomplish the
analysis but do not actually execute it.  This is mainly good for
debugging purposes.\\

\end{document}