\documentclass[12pt]{article}
\usepackage{amsmath}
%\usepackage[bottom,none]{draftcopy}
%\usepackage[light]{draftcopy}
%locate draftcopy
%\draftcopySetGray 0-1
%\draftcopyName name

%%%%%%%%%% set margins %%%%%%%%%%%%%
\addtolength{\textwidth}{1in}
\addtolength{\oddsidemargin}{-0.5in}
\addtolength{\textheight}{.75in}
\addtolength{\topmargin}{-.50in}

%%%%%%%%%%%%%%%%%%%%%%%%%%%%%%%%%%%%%%%%%%%%%%%%%%%%%%%%%%%%%%%%%
%%%%%%%%%%%%%%%%%%%%%%% begin document %%%%%%%%%%%%%%%%%%%%%%%%%%
%%%%%%%%%%%%%%%%%%%%%%%%%%%%%%%%%%%%%%%%%%%%%%%%%%%%%%%%%%%%%%%%%
\begin{document}

\begin{center}
\begin{Large}
Signal Processing and Statistical Analysis for Event-Related fMRI
\end{Large}
\begin{large}
Prepared by: Douglas N. Greve\\
February 22, 1999
\end{large}
\end{center}

\section{Introduction}

In an fMRI experiment, one or more different types of stimuli are
presented to the subject and the hemodynamic response (HDR), is
measured for a given time after presentation.  In a {\em block}
experimental design, the time between the offset of one stimulus and
the onset of the next is longer than it takes for the hemodynamic
response to regain equilibrium.  In an {\em event-related} design, one
stimulus can appear before the hemodynamic response from the previous
stimulus has decayed away.  This allows for more interesting
psychological experiments but creates overlap in the responses.  This
overlap can confound the results.  This document describes the signal
processing and statistical analysis of the fMRI signal for
event-related designs. This processing stream accounts for the overlap
in the hemodynamic response as well as compensates for the temporal
correlation in the fMRI noise.

\section{Model of the Hemodynamic Response}

The hemodynamic response {\bf at a single voxel} to the same stimulus
(or same type of stimulus) presented several times over an experiment
is assumed to be governed by following linear time-invariant system:
\begin{equation}
y(t) = x(t)*h(t) + n(t)
\label{fmrictmodel1.eqn}
\end{equation}
where $y(t)$ is the fMRI signal (ie, the observable), $x(t)$ is a
pulse train which has a value of 1 at the onset time of each
presentation and zero everywhere else\footnote{The list of the type of
stimulus presented at a certain time is the content of the
``parfile''.}, $h(t)$ is the (unknown) hemodynamic response to a
single presentation of the stimulus, $n(t)$ is additive noise
distributed $N(0,\Sigma_n)$, and $*$ is the convolution operator.
Furthermore, we assume the noise covariance matrix $\Sigma_n$ can been
factored into two components:
\begin{equation}
\Sigma_n = \sigma_n^2 C_n,
\end{equation}
where $\sigma_n^2$ is the (scalar) variance of the noise at the voxel
and $C_n$ is a symmetric, positive-definite matrix with ones on the
diagonal (the ``normalized'' covariance matrix).  It is further
assumed that, while each voxel will have its own $\sigma_n$, $C_n$
will be the same across all voxels (or at least all voxels within the
skull).  If the noise is white, then $\Sigma_n = \sigma_n^2 I$, where
$I$ is the identity matrix. When the stimulus is brief, $h(t)$ becomes
the hemodynamic impulse response.

If the stimulus sequence has different types of stimuli (or
conditions) that generate a different HDR, it is assumed that
the HDRs combine linearly according to the extension of equation
\ref{fmrictmodel1.eqn}:
\begin{equation}
y(t) = x_1(t)*h_1(t) + x_2(t)*h_2(t) + \ldots + x_{N_c}(t)*h_{N_c}(t)
+ n(t)
\label{fmrictmodel.eqn}
\end{equation}
where $N_c$ is the number of different types of stimuli,
$x_i(t)$ is the presentation sequence for stimulus type $i$, and
$h_i(t)$ is the hemodynamic response to a stimulus type $i$.

The continuous-time equation above can be converted into 
the following discrete-time, matrix model of the fMRI signal:
\begin{equation}
y(k) = X_1 h_1(k) + X_2 h_2(k) + \ldots + X_{N_c} h_{N_c}(k) + n(k),
\label{fMRIModel.eqn}
\end{equation}
where $k$ ranges from 0 to $N_{tp}$ (the number of time-points or
observations), $X_i$ is the stimulus convolution matrix (SCM) for
stimulus type $i$, and $h_i(k)$ is the continuous-time HDR sampled at
a delay $TR*k$ after the onset of a stimulus of type $i$, where $TR$
is the time-between observations.  The HDR is sampled over a time window
of $T_{HDR}$ resulting in $N_h = T_{HDR}/ TR$ samples.  This is a
moving-average (MA) or finite-impulse response (FIR) model.

$X_i$, the SCM, is an asymmetric toeplitz matrix of ones and zeros. It
has dimension $N_{tp} \times N_h$ (the number of time-points by the
number of delays) and is constructed from the stimulus sequence
$x_i(t)$.  Let $s_i(k)$ be the stimulus sequence sampled at the TR.
$s_i(k)$ is then a vector of ones and zeros.  The columns of $X_i$ are
shifted version so of $s_i(k)$. Thus, the first column of $x_i$ is
identical to $s_i(k)$, where $k$ becomes the row number.  The second
column has zero on the first row; the remainder of the rows are
assigned the values $s_i(k), 1 \le k \le N_t-1$.  The third column has
zeros on the first two rows, etc.  

Equation \ref{fMRIModel.eqn} can be further consolidated in matrix
notation:
\begin{equation}
y = X h + n,
\label{fmrix.eqn}
\end{equation}
where $X = [X_1 X_2 \ldots X_{N_c}]$, ie a horizontal concatenation
of the SCMs of the individual conditions, and $h$ is the vertical
concatenation of the individual HDRs.  $X$ has dimension $N_{tp}
\times  N_{ch}$ where $N_{ch} = N_c N_h$ is the total number of 
hemodynamic coefficients (ie, across all conditions and delays).

\section{HDR Estimation}

In Equation \ref{fmrix.eqn}, we know $X$ and $y$ and seek to estimate
$h$.  The hemodynamic response is obtained by solving
\begin{equation}
\text{min} \parallel B^{-1}_n (X h - y) \parallel_2
\end{equation}
where $B_n = C_n^\frac{1}{2}$ is the Cholesky factorization of $C_n$.
This is the {\em generalized} least squares problem, which has the
solution
\begin{equation}
h^* = (X^T C_n^{-1} X)^{-1} X^T C_n^{-1} y.
\label{hstar.eqn}
\end{equation}

The {\em estimation error} is the difference between the estimated HDR
and the actual HDR:
\begin{equation}
e_{h^*} = h - h^* = (X^T C_n^{-1} X)^{-1} X^T C_n^{-1} n
\label{esterr.eqn}
\end{equation}
This should not be confused with the {\em residual error} which is
\begin{equation}
e_{y^*} = y - y^* = y - X h^*
\label{resstar.eqn}
\end{equation}
The generalized least squares solution minimizes the estimation error
whereas the classical least squares solution minimizes the residual
error.  In the special case where the noise is white (ie, $C_n = I$),
the same $h^*$ minimizes both.

$h^*$ is a random variable and will have a covariance matrix:
\begin{equation}
\Sigma_{h^*} = E(e_{h^*} e^T_{h^*}) = \sigma_n^2 (X^T C^{-1}_n X)^{-1}
\end{equation}
Note that the RMS estimation error is equal to $trace(\Sigma_{h^*})$
which is independent of the actual value of $h$.

\section{Noise Covariance Matrix Estimation}

Estimation of the noise covariance matrix $\Sigma_n$ requires two
levels of processing.  First we assume that the noise is white,
$\Sigma_n = \sigma_n^2 I$, in which case our estimate from equation
\ref{hstar.eqn} becomes:
\begin{equation}
\bar{h} = (X^T X)^{-1} X^T y,
\label{hbar.eqn}
\end{equation}
and the residuals are
\begin{equation}
e_{\bar{y}} = y - \bar{y} = y - X \bar{h} = (I - X (X^T X)^{-1} X) n.
\label{resbar.eqn}
\end{equation}

For well counter-balanced sequences and $N_{tp}>>N_{ch}$, the elements
of the matrix $X (X^T X)^{-1} X$ are small, and so, to a first
approximation, the the residuals are a good estimate of the noise and
so we will use the residuals to estimate $C_n$.  Since $C_n$ is the
normalized covariance matrix for all voxels within the skull, we will
compute it in the following way.  First, the within skull voxels are
identified as all the voxels greater than mean voxel value computed
over space and time for a given slice over a given run.  For each
within-skull voxel, the normalized, unbiased autocorrelation of the
residuals is computed.  Normalizing forces the zero-delay coefficient
to be unity.  The global correlation function is computed by averaging
each delay coefficient across all within-skull voxels.  This global
autocorrelation function is then used to fit the parameters in the
following model:
\begin{equation}
\bar{R}_e(k) = 
\begin{cases} 
1 & k = 0 \\
(1-\alpha) \rho^k & 0 < |k| \le k_{max}\\
0 & |k| > k_{max}\\
\end{cases} 
\label{acormod.eqn}
\end{equation}
where $k_{max}$ is used to limit the delay range over which the data are
fit.  For TR of 2 seconds, $k_{max}$ is typically set to 10.  Once
$\alpha$ and $\rho$ have been computed, a ``synthetic'' autocorrelation
function is computed from equation \ref{acormod.eqn}.  $\bar{C}_n$ is
then generated as a Toeplitz matrix using the synthetic function as a
seed vector.

This is somewhat of a convoluted process, however, it is motivated
by several factors: 
\begin{itemize}
\item Noise from biological material has different spectral properties
than that from air.
\item We assume that $C_n$ is {\em not} a random variable and so it
should not depend upon how it is measured.  We approximate this by
fitting only two parameters to an average computed over a large number
of voxels so that measurement artifacts will have little effect at any
particular voxel.
\end{itemize}

We substitute this estimate of the noise covariance matrix into
the formulas for the average and covariance of the HDR:
\begin{equation}
\hat{h} = (X^T \bar{C}_{n}^{-1} X)^{-1} X^T \bar{C}^{-1}_n y.
\label{hhat.eqn}
\end{equation}
The estimation error is 
\begin{equation}
e_{\hat{h}} = h - \hat{h} = (X^T \bar{C}_{n}^{-1} X)^{-1} X^T 
   \bar{C}^{-1}_n n
\label{hhaterr.eqn}
\end{equation}

\begin{equation}
\Sigma_{\hat{h}} = \hat{\sigma}_n^2 (X^T \bar{C}_{n}^{-1} X)^{-1},
\end{equation}
where $\hat{\sigma}_n^2$ is the variance of $e_{\hat{h}}$.
Note that $\Sigma_{\hat{h}}$ can be factored into a voxel-dependent
scalar and a voxel-independent matrix.  This is important for 
practical reasons has a matrix does not need to be stored
for every voxel. 

\section{Detrending}

It is common for the fMRI signal offset to drift linearly and/or
quadratically with time for non-physiological reasons.  As this drift
can skew the results, we need a way to remove both the offset and the
offset drift.  We refer to the remove of the offset and drift as {\em
detrending}.  To implement the detrending process, we first modify the
model of the hemodynamics (equation \ref{fmrictmodel.eqn}) to include
offset and drift:
\begin{equation}
y(t) = x(t)*h_(t) + n(t) + a_0 + a_1 t + a_2 t^2 + \ldots +
a_{N_{dt}-1} t^{N_{d}t-1} +n(t),
\label{fmrictdrift.eqn}
\end{equation}
where the value $a_i$ is the scalar coefficient of the $i^{th}$ trend.
In the above equation, trends up to order $N_{dt}$ are represented,
where the first trend is a simple offset.  

In matrix notation, equation \ref{fmrictdrift.eqn} can be represented
by:
\begin{equation}
y_{pre} = X_{dt} h_{dt} + n,
\end{equation}
where $y_{pre}$ is used to indicate the fMRI signal prior to
detrending, 
\begin{equation}
X_{dt} =  
\begin{bmatrix} X & t_s^0 & t^1 & \ldots & t_s^{N_{dt}} \end{bmatrix} 
= [X D]
\end{equation}
and
\begin{equation}
h_{dt} = \begin{bmatrix} h & a_0 & a_1 & \ldots &  a_{N_{dt}}
\end{bmatrix} = 
\begin{bmatrix}
h & a
\end{bmatrix}
\end{equation}
and $t_s$ is a column vector of sample times ,ie, 
$t_s = \begin{bmatrix}
  0 & TR & 2TR & \ldots & (N_{tp}-1)TR
\end{bmatrix}$.
The hemodynamic responses are fit
simultaneously with the trends (assuming white noise).
\begin{equation}
\Tilde{h}_{dt} = (X_{dt}^T X_{dt})^{-1} X_{dt}^T y_{pre} 
\end{equation}
The trend can now be removed by:
\begin{equation}
y = y_{pre} - D \Tilde{h}_{dt}
\label{detrend.eqn}
\end{equation}
$y$ is now the de-trended fMRI signal, and processing can proceed as
indicated above with the one caveat that the degrees of freedom must
be reduced by $N_{dt}$.

\section{Processing Stream Summary}

For multiple runs within a single session, the analysis proceeds 
according to the following steps.  Each slice is processed separately
and independently.
\begin{enumerate}
\item Compute stimulus convolution matrix for each run, $X_r$.
\item Detrend each run (equation \ref{detrend.eqn}).
\item Estimate the hemodynamic response assuming white noise
over the entire session according to the formula: 
\begin{equation}
\bar{h}_S = \left(\sum_{r=1}^{N_r} {X_r^T X_r}\right)^{-1} 
\left(\sum_{r=1}^{N_r} {X_r^T y_r}\right)
\label{hsbar.eqn}
\end{equation}
\item Compute the residual errors for each run
\begin{equation}
e_{\bar{y},r} = y_r - X_r \bar{h}_S
\label{resbarrun.eqn}
\end{equation}
\item Compute the estimated noise covariance matricies, 
$\bar{C_{n,r}}$, for each run.

\item Recompute the session hemodynamic response estimate 
\begin{equation}
\hat{h}_s = \left(\sum_{r=1}^{N_r} {X_r^T \bar{C}_{n,r}^{-1} X_r}\right)^{-1} 
       \left(\sum_{r=1}^{N_r} {X_r^T \bar{C}_{n,r}^{-1} y_r} \right),
\label{hshat.eqn}
\end{equation}

\item Recompute the residual error at each voxel:
\begin{equation}
e_{\hat{h}_r} = y_r - X_r \hat{h}_s
\end{equation}

\item Compute the variance of the residual over the entire session for
each voxel:
\begin{equation}
\hat{\sigma}_{n,s}^2 = \frac{\sum_{r=1}^{N_r} 
    e_{\hat{h}_r}^T \bar{C}_{n,r}^{-1} e_{\hat{h}_r}}{N_{dof}},
\end{equation}
where the degrees of freedom, $N_{dof} = N_r * N_{tp} - N_{ch}$.

\item Compute the session hemodynamic covariance matrix:
\begin{equation}
\Sigma_{\hat{h}_s} = \hat{\sigma}_{n,s}^2 
  \left(\sum_{r=1}^{N_r} {X_r^T  \bar{C}_{n,r}^{-1} X_r}\right)^{-1} =
 \sigma^2_{\hat{h}_s} C_{\hat{h}_s} 
\end{equation}
where $\Sigma_{\hat{h}_s}$ can be factored into a voxel-dependent
scalar and a voxel-independent matrix.

\end{enumerate}

\section{Statistical Inference}

The goal of the fMRI experiment is to determine whether a location in
the brain is becoming active in response to a particular stimulus type
or whether one type activates a particular region more than another.
Tests can also be restricted by range of post-stimulus delay.  For
example, the null hypothesis could be that none of the conditions at
any delay generate a hemodynamic response that is significantly
different than zero:
\begin{equation}
H_0: \| \hat{h} \| = 0
\label{hoall.eqn}
\end{equation}
This can be tested using an F-test:
\begin{equation}
F(N_{ch}, N_{dof}) = \frac{\hat{h}^T C_{\hat{h}}^{-1}\hat{h}}
                          { N_{ch} \sigma^2_{\hat{h}}}
\label{ftestall.eqn}
\end{equation}
In general, one may want to restrict the test to a subset of delays
and conditions or combinations of conditions.  For example,
$H_0: \| \hat{h}_1 - \hat{h}_2 \| = 0$ or 
$H_0: \| \hat{h}_3(3:6) \| = 0$.  Mathematically, this is equivalent
to multiplying $\hat{h}$ by a matrix $R$ and testing the norm of the
result.  
\begin{equation}
q = R \hat{h}, H_0: \| q \| = 0
\end{equation}
with the corresponding F test
\begin{equation}
F_q(N_{q}, N_{dof}) = \frac{q^T (R C_{\hat{h}} R^T)^{-1} q}
{N_{q} \sigma^2_{\hat{h}}},
\label{ftestrm.eqn}
\end{equation} 
where $N_q$ is the number of rows in $R$. The restriction matrix, $R$,
is typically a matrix of $\pm 1$'s and 0's.  The number of columns of
$R$ must be equal to $N_{ch}$.  In the example given in equations
\ref{hoall.eqn} and \ref{ftestall.eqn}, $R$ would simply be an
identity matrix of dimension $N_{ch}$.  When $R$ is a vector, then the
F test of equation \ref{ftestrm.eqn} reduces to a t test.  Note that
equation \ref{ftestrm.eqn} can again be factored into a
voxel-dependent scalar and a voxel-independent matrix.


\section{Table of Variable Names}

\begin{table}
\begin{center}
\begin{tabular}{|l|l|}\hline
$y$   & fMRI signal for a single voxel \\ \hline 
$y_{pre}$   & fMRI signal for a single voxel before detrending\\ \hline 
$X$   & Stimulus convolution matrix (SCM)\\ \hline 
$D$   & Trend Matrix\\ \hline 
$X_{dt}$   & Stimulus convolution matrix with trend matrix included\\ \hline 
$h$   & ``True'' HDR. \\ \hline 
$h_{dt}$   & ``True'' HDR with trend coefficients \\ \hline 
$n$ & Zero-mean gaussian noise  \\ \hline 
$\Sigma_n$  & Noise covariance matrix \\ \hline
$C_n$  & Normalized noise covariance matrix (voxel-independent) \\ \hline
$\sigma_n$  & Noise variance (voxel-dependent) \\ \hline
$h^*$ & best fit HDR when $C_n$  is known \\ \hline
$\bar{h}$ & best fit HDR when the noise is white \\ \hline
$\hat{h}$  & best fit HDR when $C_n$  is estimated from residuals.\\ \hline
$\Tilde{h}_{dt}$   & best fit HDR with trend coefficients \\ \hline 
$e_{h^*}$, $e_{\bar{h}}$, $e_{\hat{h}}$ & estimation errors. \\ \hline 
$\Sigma_{\hat{h}}$  &  Covariance matrix of  $\hat{h}$\\ \hline
$C_{\hat{h}}$  &  Normalized covariance matrix of  $\hat{h}$
           (voxel-independent) \\ \hline
$\sigma_{\hat{h}}$  & Variance of  $\hat{h}$ (voxel-dependent) \\ \hline
$R_e$ & Autocorrelation of residual errors \\ \hline
$\alpha$,$\rho$ & parameters used to fit $R_e$. \\ \hline
$k_{max}$ & maximum number of coefficients of $R_e$ to fit. \\ \hline
$R$ & Restriction Matrix \\ \hline
$q$ & Statistical test vector (equals $R h$) \\ \hline
$N_{tp}$ & Number of time points (ie, scans) per run.\\ \hline
$N_h$ & Number of parameters (per stimulus) estimated in HDR.\\ \hline
$N_c$ & Number of stimulus conditions.\\ \hline
$N_{ch}$ & Total number of parameters estimated ($N_c * N_h$)\\ \hline 
$N_{dt}$ & Detrending order.\\ \hline 
$N_{r}$ & Number of runs.\\ \hline 
$N_q$ & Number of rows in the restriction matrix.\\ \hline
$TR$ & Time between observations.\\ \hline 
$T_{HDR}$ & Time window over which to estimate the HDR ($N_h * TR$).\\ \hline
$N_S$ & Number of sessions (or subjects)\\ \hline
$N_R$ & Number of runs per session    \\ \hline
\end{tabular}
\end{center}
\end{table}


\end{document}
%%%%%%%%%%%%%%%%%%%%%%%%%%%%%%%%%%%%%%%%%%%%%%%%%%%%%%%%%%%%%%%%%
%%%%%%%%%%%%%%%%%%%%%%% end document %%%%%%%%%%%%%%%%%%%%%%%%%%%%
%%%%%%%%%%%%%%%%%%%%%%%%%%%%%%%%%%%%%%%%%%%%%%%%%%%%%%%%%%%%%%%%%




